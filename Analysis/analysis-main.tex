\documentclass{article}

\usepackage{preamble}

\title{Analysis Qual Solutions}
\author{Madison Phelps \thanks{Oregon State University, phelpmad@oregonstate.edu}}
\date{\today}

\begin{document}

\maketitle

\thispagestyle{empty}

\break

\tableofcontents

\thispagestyle{empty}

\break

\section{Contraction Mapping Theorem}

\begin{problem}{Fall 2020 \# 1} Let 
	\[F(x) = \frac{1}{2}\left(x + \frac{a}{x}\right)\]
		for all $x>0$, where $a>0$ is constant. 
		This problem is concerned with sequences $\{x_n\}_{n=0}^\infty$ 
		defined by $x_{n+1} = F(x_n)$ for $n\geq 0$, with $x_0$ chosen arbitrarily.
	\begin{enumerate}
		\item[(a)] Prove that there exists a closed interval $I$ with positive finite length such that $F$ maps $I$ into $I$.
		
		\item[(b)] Prove that for every $x_0\in I$, the sequence $\{x_n\}_{n=0}^\infty$ converges to a limit 
			that is independent of the choice of $x_0$. What is this limit?
	\end{enumerate}
\end{problem}


\textbf{Solution:} (a)

Define $I = [\sqrt{a}, b]$, where $a>0$ is given and choose $b\in\R$ such that $b>\sqrt{a}>0$. To show that $F(I)\subseteq I$, we first compute the solution to $F'(x) = \frac{1}{2}\left(1 - \frac{a}{x^2}\right) = 0$ to find that $F$ has a minimum at $x = \sqrt{a}$ (since $x>0$). Also, observe $F$ has a fixed point at $x = \sqrt{a}$ because
	\[F(\sqrt{a}) = \frac{1}{2}\left(\sqrt{a} + \frac{a}{\sqrt{a}}\right) = \sqrt{a}.\]
	
Thus, $F$ is bounded below by $x = \sqrt{a}$. Since $0<\sqrt{a}<b$, we use the two inequalities $\frac{a}{b}< \sqrt{a}$ and $\sqrt{a}+b < 2b$ to conclude that
	\[F(b) = \frac{1}{2}\left(b + \frac{a}{b}\right) < \frac{1}{2}(\sqrt{a} + b) < \frac{1}{2}(2b) = b. \]
That is, for any $b\in\R$ with $\sqrt{a}<b$, $F$ is bounded above by $b$. Therefore, for all $x\in I$, $\sqrt{a} \leq F(x) < b$, and so, $F(I) \subseteq I$.


\textbf{Solution:} (b) 

To prove the statement, we will show that $F$ is a contraction mapping on $I$ with constant $\alpha = \frac{1}{2} \in [0, 1)$ and apply the Banach fixed point theorem which states:

If $f: X\to X$ is a mapping on a complete metric space $X$ such that there exists $\alpha \in [0,1)$ such that 
\[|f(x) - f(y)|\leq \alpha |x-y|\]
for all $x,y,\in X$, then $f$ has a unique fixed point. Moreover, for any $x_0\in X$, the sequence of functional iterates $x_0, f(x_0), f(f(x_0)),\dots$ converges to the fixed point in $X$. 

Observe that in order for $F$ to be a contraction mapping we must show that 
	\begin{equation}\label{eqn:fall2020-contraction}
		\left|1 - \frac{a}{xy}\right| < 1 \text{ for all } x,y\in I.
	\end{equation}
Suppose $x = y = \sqrt{a}$, then $0 = \left|1 - \frac{a}{a} \right| < 1$. Conversely, if $x = y = b$, then $b > \sqrt{a}$ implies that $\frac{a}{b^2} < 1$ and so, $0 = \left|1 - \frac{a}{b^2} \right| < 1$. Lastly, let $x,y\in I$ be distinct and such that $\sqrt{a} < x,y < b$s. Then, 
	\begin{equation}\label{eqn:fall2020-inequality}
		\frac{1}{a} > \frac{1}{xy} \;\; \text{ and } \;\; \frac{1}{b^2} < \frac{1}{xy}
	\end{equation}
imply that 
\[ 0 \leq \left|1 - \frac{a}{a}\right| < \left|1 - \frac{a}{xy}\right|  \leq \left|1 - \frac{a}{b^2}\right| < 1 \]
for all $x,y\in I$. Using inequality (\ref{eqn:fall2020-contraction}) for any $x,y\in I$
	\begin{align}
		|F(x) - F(y)| & =\frac{1}{2} \left|x +\frac{a}{x} - \left(y +\frac{a}{y}\right)\right|\\
				  & = \frac{|x-y|}{2}\left|1 - \frac{a}{xy}\right|\\
				  & < \frac{1}{2}|x-y|.
	\end{align}
Since $I\subseteq\R$ is a closed and bounded set in $\R$, $I$ is complete and because $F$ is a contraction mapping (from above and by part (a)) with contraction constant $\alpha = {1}{2}$, the Banach fixed point theorem states that: (1) $F$ has a unique fixed point, and (2) for any $x_0\in I$ the sequence $\{x_n\}_{n = 0}^\infty$ converges to the fixed point and is independent of the choice of $x_0$.\\ 

\hrule 

\textbf{Notes:} The above question is regarding the Banach fixed point theorem and the consequence that the sequence of functional iterates converges to the fixed point no matter what point in the complete space you choose. However, if the space is not complete and you find a complete space within the set for which $f$ is defined in, eventually the sequence will be inside of the complete space. Of course, you need to show that $f$ is a contraction mapping. But, only finitely many terms will be outside of the complete set. Then, there will be a point where the Banach fixed point theorem kicks in and we can guarantee that the sequence of functional iterates converges to to the unique fixed point of $f$. 

Another thing with this problem is that it utilizes working with inequalities and boils down to manipulating the inequality $b > \sqrt{a}$ in many different ways for some fixed $a>0$.\\ 

\hrule \vspace{2pt}
\hrule 

\break 

\textbf{Solution:} (a)


\end{document}