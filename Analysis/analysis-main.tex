\documentclass{article}

\usepackage{preamble}

\title{Analysis Qual Solutions}
\author{Madison Phelps \thanks{Oregon State University, phelpmad@oregonstate.edu}}
\date{\today}

\begin{document}

\maketitle

\thispagestyle{empty}

\break

\tableofcontents

\thispagestyle{empty}

\break

\section{Contraction Mapping Theorem}

\subsection{Theorems and Definitions}

\begin{theorem}[Banach Fixed Point Theorem]\label{thm:banach-fixed-point}
	If $f: X\to X$ is a mapping on a complete metric space $X$ such that there exists $\alpha \in [0,1)$ such that 
	\[|f(x) - f(y)|\leq \alpha |x-y|\]
	for all $x,y,\in X$, then $f$ has a unique fixed point. Moreover, for any $x_0\in X$, the sequence of functional iterates 
	$x_0, f(x_0), f(f(x_0)),\dots$ converges to the fixed point in $X$. 
\end{theorem}

\break

\subsection{Problems}

\begin{problem}{Fall 2020 \# 1} Let 
	\[F(x) = \frac{1}{2}\left(x + \frac{a}{x}\right)\]
		for all $x>0$, where $a>0$ is constant. 
		This problem is concerned with sequences $\{x_n\}_{n=0}^\infty$ 
		defined by $x_{n+1} = F(x_n)$ for $n\geq 0$, with $x_0$ chosen arbitrarily.
	\begin{enumerate}
		\item[(a)] Prove that there exists a closed interval $I$ with positive finite length such that $F$ maps $I$ into $I$.
		
		\item[(b)] Prove that for every $x_0\in I$, the sequence $\{x_n\}_{n=0}^\infty$ converges to a limit 
			that is independent of the choice of $x_0$. What is this limit?
	\end{enumerate}
\end{problem}


\textbf{Solution:} (a)

Define $I = [\sqrt{a}, b]$, where $a>0$ is given and choose $b\in\R$ such that $b>\sqrt{a}>0$. To show that $F(I)\subseteq I$, we first compute the solution to $F'(x) = \frac{1}{2}\left(1 - \frac{a}{x^2}\right) = 0$ to find that $F$ has a minimum at $x = \sqrt{a}$ (since $x>0$). Also, observe $F$ has a fixed point at $x = \sqrt{a}$ because
	\[F(\sqrt{a}) = \frac{1}{2}\left(\sqrt{a} + \frac{a}{\sqrt{a}}\right) = \sqrt{a}.\]
	
Thus, $F$ is bounded below by $x = \sqrt{a}$. Since $0<\sqrt{a}<b$, we use the two inequalities $\frac{a}{b}< \sqrt{a}$ and $\sqrt{a}+b < 2b$ to conclude that
	\[F(b) = \frac{1}{2}\left(b + \frac{a}{b}\right) < \frac{1}{2}(\sqrt{a} + b) < \frac{1}{2}(2b) = b. \]
That is, for any $b\in\R$ with $\sqrt{a}<b$, $F$ is bounded above by $b$. Therefore, for all $x\in I$, $\sqrt{a} \leq F(x) < b$, and so, $F(I) \subseteq I$.


\textbf{Solution:} (b) 

To prove the statement, we will show that $F$ is a contraction mapping on $I$ with constant $\alpha = \frac{1}{2} \in [0, 1)$ and apply the Banach fixed point theorem which states:

If $f: X\to X$ is a mapping on a complete metric space $X$ such that there exists $\alpha \in [0,1)$ such that 
\[|f(x) - f(y)|\leq \alpha |x-y|\]
for all $x,y,\in X$, then $f$ has a unique fixed point. Moreover, for any $x_0\in X$, the sequence of functional iterates $x_0, f(x_0), f(f(x_0)),\dots$ converges to the fixed point in $X$. 

Observe that in order for $F$ to be a contraction mapping we must show that 
	\begin{equation}\label{eqn:fall2020-contraction}
		\left|1 - \frac{a}{xy}\right| < 1 \text{ for all } x,y\in I.
	\end{equation}
Suppose $x = y = \sqrt{a}$, then $0 = \left|1 - \frac{a}{a} \right| < 1$. Conversely, if $x = y = b$, then $b > \sqrt{a}$ implies that $\frac{a}{b^2} < 1$ and so, $0 = \left|1 - \frac{a}{b^2} \right| < 1$. Lastly, let $x,y\in I$ be distinct and such that $\sqrt{a} < x,y < b$s. Then, 
	\begin{equation}\label{eqn:fall2020-inequality}
		\frac{1}{a} > \frac{1}{xy} \;\; \text{ and } \;\; \frac{1}{b^2} < \frac{1}{xy}
	\end{equation}
imply that 
\[ 0 \leq \left|1 - \frac{a}{a}\right| < \left|1 - \frac{a}{xy}\right|  \leq \left|1 - \frac{a}{b^2}\right| < 1 \]
for all $x,y\in I$. Using inequality (\ref{eqn:fall2020-contraction}) for any $x,y\in I$
	\begin{align}
		|F(x) - F(y)| & =\frac{1}{2} \left|x +\frac{a}{x} - \left(y +\frac{a}{y}\right)\right|\\
				  & = \frac{|x-y|}{2}\left|1 - \frac{a}{xy}\right|\\
				  & < \frac{1}{2}|x-y|.
	\end{align}
Since $I\subseteq\R$ is a closed and bounded set in $\R$, $I$ is complete and because $F$ is a contraction mapping (from above and by part (a)) with contraction constant $\alpha = {1}{2}$, the Banach fixed point theorem states that: (1) $F$ has a unique fixed point, and (2) for any $x_0\in I$ the sequence $\{x_n\}_{n = 0}^\infty$ converges to the fixed point and is independent of the choice of $x_0$.\\ 

\hrule 

\textbf{Notes:} The above question is regarding the Banach fixed point theorem and the consequence that the sequence of functional iterates converges to the fixed point no matter what point in the complete space you choose. However, if the space is not complete and you find a complete space within the set for which $f$ is defined in, eventually the sequence will be inside of the complete space. Of course, you need to show that $f$ is a contraction mapping. But, only finitely many terms will be outside of the complete set. Then, there will be a point where the Banach fixed point theorem kicks in and we can guarantee that the sequence of functional iterates converges to to the unique fixed point of $f$. 

Another thing with this problem is that it utilizes working with inequalities and boils down to manipulating the inequality $b > \sqrt{a}$ in many different ways for some fixed $a>0$.\\ 

\hrule \vspace{2pt}
\hrule 

\break 

\begin{problem}{Jordan Qual Prep Week 1 (\& 7), \# 3} Let $f:\R \to \R$ be a Lipshitz continuous function. That is, 
	\[\frac{ |f(x)- f(y)| }{|x-y|} \leq L \quad \text{ for some constant $L$ and for all $x\neq y$.}\]
	\begin{enumerate}
		\item[(a)] Prove that $f$ is uniformly continuous.
		\item[(b)] Prove that if $L<1$, then $f$ has a fixed point. 
		\item[(c)] Let $x_{n+1} = \frac{ 1 }{2( 1+ x_n)}$ with $x_0= 0$. Prove that the sequence $\{x_n\}$ is convergent.
	\end{enumerate}
\end{problem}

\textbf{Solution:} (a)

Let $\varepsilon > 0$, choose $\delta = \frac{\varepsilon}{L}>0$ and let $x\in\R$ be arbitrary. Then, for any $y\in \R$ such that $|x-y|<\delta$,
	\[ |f(x) - f(y)| \leq L|x-y| < \varepsilon.\]
Therefore, $f$ is uniformly continuous because $\delta>0$ is independent of $x\in\R$. 

\textbf{Solution:} (b)

Since $\R$ is complete, $L<1$, and $f(\R)\subseteq \R$, we conclude that $f$ is a contraction mapping on $\R$. By the Banach fixed point theorem (see Theorem \ref{thm:banach-fixed-point}), $f$ has a unique fixed point.

\textbf{Solution:} (c)

Let $t\in\R$. Observe that for any $t\in \R$, $[t,\infty)\subset \R$ is a closed subspace of $\R$ because the complement of $X_t$ is open. Since any closed subset of a complete space, is also complete, we conclude that $X_t$ is complete. 

Define $f(x) = \frac{1}{2(1+x)}$. We will show that for any $t\in \R$ such that $0 \leq t \leq \frac{\sqrt{3} - 1}{2}$, $f$ is a contraction mapping on a complete space defined by $X_t = [t, \infty)$, then conclude that $f$ has a unique fixed point and the sequence of functional iterates for any $x_0\in X_t$ converges to the fixed point. 

First, $x = \frac{\sqrt{3}-1}{2}$ is a fixed point of $f$ by solving $f(x) = x$ and using the quadratic formula on $x^2 + x -\frac{1}{2} = 0$. Now suppose that $t >  \frac{\sqrt{3}-1}{2}$. Then, 
	\[ t >  \frac{\sqrt{3}-1}{2} = f( \frac{\sqrt{3}-1}{2}) = \frac{1}{2(1+ \frac{\sqrt{3}-1}{2})} > \frac{1}{2(1+t)} = f(t),\]
and $f(t) < t$ implies that $f(t) \not\in X_t$ for any $t$ larger than the fixed point. Next, $f$ is bounded below by 0 because $f$ is defined on $(-1,\infty)$ and for any $x>-1$, $f(x) > 0$. To see that $t = 0$ is also true, we compute $f(0) = \frac{1}{2} \in X_0 = [0,\infty)$. Thus, for any $0 \leq t \leq \frac{\sqrt{3} - 1}{2}$, $f(X_t) \subseteq X_t$. 

Next, we will show that $f$ is a contraction mapping. Fix $t$, where $0\leq t\leq \frac{\sqrt{3} - 1}{2}$. Then for any distinct $x,y\in X_t$, assume $0\leq t \leq  x < y$. Then, we have $1+y > 1+t  > 1$ and $1+x \leq 1+ t \leq 1$ implies that 
	\[ \frac{1}{(1+x)(1+y)}< \frac{1}{(1+t)^2} \leq 1.\]
Because $x$ and $y$ are distinct, $\frac{1}{(1+x)(1+y)}<1$ for all $x,y\in X_t$. So, observe that
	\begin{align}
		|f(x) - f(y)| & = \frac{1}{2}\left| \frac{1}{(1+x)} - \left(\frac{1}{(1+y)}\right)\right| \\
				& = \frac{|x-y|}{2} \left| \frac{1}{(1+x)(1+y)}\right|\\
				& < \frac{1}{2}|x-y|
	\end{align}
which shows that $f$ is a contraction mapping on a complete space with constant $\frac{1}{2}\in[0,1)$. Therefore, by Banach fixed point theorem by choosing $t = 0$, we can conclude that the sequence defined by $f(x_n) = x_{n+1} = \frac{1}{2(1+x_n)}$ with $x_0 = 0 \in X_t$ converges.\\

\hrule 

\textbf{Notes:} Well, this is the problem for you if you want to conquer inequalities. The goal is to think smarter and not harder. When you catch yourself doing the thing that is not ``obvious" you should revert back to the original idea you are trying to prove. Also, keep track at each step that your inequalities are all facing the correct way at each step in your scratch work. It is very easy to make $<$ a $>$ and then sit there for hours thinking about how stupid inequalities are. Also, computing the fixed point doesn't seem as obvious that it is a fixed point because 
	\[f((\sqrt{3}-1))/2) = 1/(\sqrt{3} + 1)\]
meaning that for someone that is not good at numbers (which is me) this doesn't make sense. However, in cases like these you should think to multiple by the conjugate. Then, remember the difference of two squares formula of 
	\[(a-b)(a+b) = a^2 - b^2\]
that really comes up a lot on this test.

In a nutshell, make sure that you have all of the inequalities worked out before you write the problem. Make sure you know how you want to write it up before you write it. There is no time to write and rewrite. Also, if you know what you are doing, the writing is easy and will not come across as someone that is confused.\\

\hrule \vspace{2pt}
\hrule 

\break

\begin{problem}{\#5, Spring 2021} Let $(a_n)$, $(\beta_n)$, $(\gamma_n)\in \ell_\infty$ be fixed sequences. For a sequence $(b_n)$ define 
	\[ T(b_n) = a_n + \beta_n b_{n+1} + \gamma_n b_{n+2}, \quad n\in \N.\]
\begin{enumerate}
	\item[(a)] Prove that $T$ is a well-defined continuous mapping from $\ell_\infty$ to $\ell_\infty$.
	\item[(b)] Prove that if $\norm{|\beta_n| + |\gamma_n|}_\infty < 1$, then $T$ has a fixed point. (Here $(|\beta_n|)$, 
			$(|\gamma_n|)$ are the sequences of the absolute values.)
	\item[(c)] Let $\alpha, \beta, \gamma\in\R$ such that $|\beta| + |\gamma| < |\alpha|$. Prove that for any $(a_n)\in \ell_\infty$
			there exists $(b_n)\in\ell_\infty$ such that for all $n\in \N$
				\[a_n = \alpha b_n + \beta b_{n+1} + \gamma b_{n+2}.\]
			Also show that the result fails if the assumption $|\beta| + |\gamma| < |\alpha|$ is removed. 
\end{enumerate}
\end{problem}

\textbf{Solution:} (a) 

Let $\varepsilon>0$ be given. Since $(\beta_n)$, $(\gamma_n)\in \ell_\infty$ are fixed sequences, set $\delta = \frac{\varepsilon}{\norm{|\beta_n| + |\gamma_n|}_\infty}$


\break

\section{Fundamental Theorem of Calculus Trick}

\subsection{Convergence and Completeness}

\subsubsection{Definitions and Theorems}

\subsubsection{Problems}

\begin{problem}{\#1, Fall 2021} Let $X = \{f \in C[0,1] : f(0) = 0\}$. For each $f\in X$, let 
	\[\norm{ f } = \max\{ |f'(x)| : x\in [0,1]\}.\]
Note that the norm of $f$ is defined in terms of the \textit{derivative} of $f$. 
\begin{enumerate}
	\item[(a)] Show that if the definition of $X$ is changed by deleting the condition, $f(0) = 0$, 
		then $\norm{\cdot}$ is not a norm. 
	\item[(b)] Show that if a sequence $\{f_n\}_{n=1}^\infty$ converges in $X$ with respect to the norm $\norm{\cdot}$,
		then the sequence $\{f_n\}_{n=1}^\infty$ converges uniformly on $[0,1]$.
	\item[(c)] Show that the space $X$ is complete with respect to the norm defined above.
\end{enumerate}

\end{problem}

\textbf{Solution:} (a) 

Choose $f(x) = 1\in C^1[0,1]$ for all $x\in [0,1]$. We compute, and find that $\norm{f} = 0$ but $f\neq0$ on $[0,1]$.
Therefore, if the condition that $f(0) = 0$ is removed, then $\norm{\cdot}$ does not define a norm.

\textbf{Solution:} (b) 

Let $\varepsilon > 0$ be given. By the Fundamental Theorem of Calculus, write 
	\begin{equation}\label{eqn:fall2021-1-ftc}
	 	f_n(x) = \int_0^x f'_n(t) dt \quad \text{ for all } x\in [0,1] \text{ and every } n\in \N,
	 \end{equation}
and similarly, for $f\in X$. Since the sequence $\{f_n\}_{n=1}^\infty$ converges in $X$, there exists $n\in \N$ 
such that for all $n\geq N$, $\norm{f_n - f}<\varepsilon$. Let $x\in[0,1]$ be arbitrary and consider the following, 
	\begin{align*}
		|f_n(x) - f(x)| & = \left| \int_0^x f'_n(t) dt - \int_0^x f(t) dt \right|\\
				    & \leq \int_0^x \left| f'_n(t) - f'(t) \right| dt\\
				    & \leq \int_0^x \norm{f_n - f} dt\\
				    & < \varepsilon \cdot x
	\end{align*}
By taking the supremum over all $x\in [0,1]$, we have that $\sup_{x\in [0,1]} |f_n(x) - f(x)| < \varepsilon$ for all $x\in [0,1]$.
That is, $f_n$ converges uniformly to $f$ on $[0,1]$.

\textbf{Solution:} (c)

Let $\varepsilon > 0$ be given and let $\{ f_n \}_{n=1}^\infty$ be an arbitrary Cauchy sequence in $X$. 
Then there exists $N\in \N$ such that for all $n,m\geq N$,
	\[ | f'_n(x) - f'_m(x) | \leq \max_{x\in [0,1]}\{| f'_n(x) - f'_m(x) |\} < \varepsilon. \]
Since the above inequality holds for all $x\in [0,1]$, we conclude that the sequence $\{ f'_n\}_{n=1}^\infty$ is uniformly
Cauchy on $[0,1]$. A sequence is uniformly Cauchy if and only if the sequence converges uniformly, and, moreover, 
the sequence converges to a continuous limit. So, 
	\begin{equation}\label{eqn:fall2021-1-der-unif}
		 \lim_{n\to \infty} f'_n(x) = g(x) \quad \text{ for some } g \in C[0,1]
	\end{equation}

Define $f(x) = \lim_{n\to\infty}f_n(x)$ for each $x\in[0,1]$. By the Fundamental Theorem of Calculus, (as defined by equation (\ref{eqn:fall2021-1-ftc}) we may write 
	\begin{equation*}
	 f_n(x) = \int_0^x f'_n(t) dt \quad \text{ for all } x\in [0,1] \text{ and every } n\in \N.
	 \end{equation*}
Now observe that,
	\[f(x) = \lim_{n \to \infty} f_n(x) = \lim_{n \to \infty} \int_0^x f'_n(t) dt = \int_0^x \lim_{n\to \infty} f'_n(t) dt = \int_0^x g(t) dt\]
(note that the above equalities come from our definition of $f$, equations (\ref{eqn:fall2021-1-der-unif}) and 
(\ref{eqn:fall2021-1-ftc}), and that if a function converges uniformly then we can interchange the integral and limits). 
Since $g\in C[0,1]$ and $\int_0^x g(t) dt$ is continuous, $f\in C[0,1]$ and it follows that $f'(x) = g(x) \in C[0,1]$, i.e.,
 $f\in C^1[0,1]$. Lastly, because $f_n\in X$ for all $n\in \N$, $f(0) = \lim_{n\to\infty} f_n(0) = 0$. 

Therefore, $f\in X$ and $X$ is complete with respect to the $\norm{\cdot}$ defined above.\\

\hrule

\textbf{Notes:} Well, this was a beast of a problem and it combined a lot of different facts together. In addition, it has you write the convergence of sequence using the limit definition and also use the the epsilon limit definition of a convergent sequence. One important thing is that converging uniformly is equivalent to uniformly Cauchy in the sup-norm. And, even more, the limit of a sequence of functions that converges uniformly on some space is continuous. In the last part we were kind of clever by writing $f_n$ using the derivative and the fundamental theorem of calculus. 

One nice thing that you did is that you wrote down your ideas and then wrote up the solution. Go you.

Next, the last part kind of hinges on the theorem that states: if you have a sequence of real valued functions each of which is continuously differentiable and that the sequence of the derivatives converges uniformly to some continuous function and if the sequence of functions converges point wise for some $x_0\in [a,b]$, then the sequence of real valued functions converges uniformly and the sequence of derivatives converges to the derivative of the sequence of functions.\\

\hrule\vspace{2pt}
\hrule

\break

\section{Arzeli-Ascoli Theorem}

\subsection{Problems}

\begin{problem}{\#6, Fall 2019} Let $C[0,1] = \{f: [0,1]\to\R\}$ denote the space of all continuous functions. You can use that this is a complete metric space with respect to the sup-norm, $\norm{f} = \sup\{|f(x)|: x\in[0,1]\}$. Define
	\[ \F = \{ f\in C[0,1] : |f(x) - f(y)| \leq |x-y|, \text{ for all } x,y\in[0,1]\}\]
\begin{enumerate}
	\item[(a)] Show that $\F$ is closed, but not compact in $C[0,1]$.
	\item[(b)] Show that 
		\[ \F_1 = \{ f\in \F : \int_0^1 f^2(x) dx = 1\} \]
	is compact in $C[0,1]$.
\end{enumerate}
\end{problem}

\textbf{Solution:} (a) 

Define $f_n(x) = n$ for $n\in \N$. Then, $f_n\in \F$ because $|f_n(x) - f_n(y)| = 0 \leq |x-y|$ for all $n\in \N$. However, as $n\to\infty$, $f_n(x) \to \infty$ implying that $\F$ is not bounded and therefore, is not totally bounded and is not compact as a result. \footnote{Also note that you can choose $f_n(x) = n+x$. Here we have created a sequence of functions in the set that is unbounded. Hence, the sequence does not have a Cauchy subsequence because any two functions of our set will differ by 1. A set is totally bounded if and only if every sequence in the space has a Cauchy subsequence. So, if we choose our definition of compact as "Complete and Totally Bounded", then $\F$ is not compact because $\F$ is not totally bounded.}

Let $\varepsilon > 0$ and let $\{f_n\}_{n=1}^\infty$ be an arbitrary convergent sequence in $\F$ that converges to $f\in C[0,1]$ (since $C[0,1]$ is complete). Then there exists some $N\in\N$ such that for all $n\geq N$, 
	\[ |f_n(x) - f(x)| \leq \sup_{x\in [0,1]}\{|f_n(x) - f(x)|\} < \frac{\varepsilon}{2}\quad \text{ for all } x\in [0,1].\]
Observe,
	\begin{align*}
		|f(x) - f(y)| & \leq |f_n(x) - f(x)| + |f_n(x) - f_n(y)| + |f_n(y) - f(y)|\\
				& < \frac{\varepsilon}{2} + |x-y| + \frac{\varepsilon}{2}\\
				& = \varepsilon + |x-y|
	\end{align*}
holds for all $\varepsilon > 0$ and as $\varepsilon \to 0$, $|f(x) - f(y)| < |x-y|$ which means that $f\in \F$. Therefore, $f$ contains all its limit points and $\F$ is closed.

\textbf{Solution:} (b) 

We will show that $\F_1$ is closed, equicontinuous, and uniformly bounded.

\begin{enumerate}
	\item Let $\{f_n\}_{n=1}^\infty$ be an arbitrary convergent sequence in $\F_1$ that converges to $f$. Since 
		$\F_1\subseteq C[0,1]$ and $C[0,1]$ is complete under the sup-norm, convergence of this sequence 
		implies that $f_n$ converges uniformly to $f$ on $[0,1]$. Since each $f_n\in \F_1$, we see that $F_1$ is
		 closed because
			\[ 1 = \lim_{n\to\infty} \int_0^1 f_n^2(x) dx = \int_0^1 \lim_{n\to\infty} f_n^2(x) dx = \int_0^1 f^2(x) dx,\]
		that is $f\in \F_1$. 
	\item Let $\varepsilon>0$ be given. Choose $\delta = \varepsilon > 0$. Then, for any $f\in \F_1 \subset \F$, 
		\[ |f(x) - f(y)| \leq |x-y| < \delta = \varepsilon.\]
		So, $\F_1$ is equicontinuous.
		
	\item Let $f\in \F_1\subset \F$ be arbitrary, then observe
			\[ |f(x) - f(0)| \leq |x-0| = |x|\]
		and expanding the inequality (as $-|x| \leq f(x) - f(0) \leq |x|$), repackaging, and using the triangle inequality
		 we have that 
			\[|f(x)| \leq |f(0)| + |x|\]
		for all $f\in \F_1$. Suppose $f(0) = \pm 2$. Since $f$ is Lipschitz with constant $1$ at best we can bound $f$
		above by the line $g_1(x) = 2-x$ and below by the line $g_2(x) = -2+x$. Then, by squaring and integrating 
		the inequality $|f(x)| \leq |f(0)| + |x|$, we have
		 	\[ 1 = \int_0^1 f^2(x) dx < \int_0^1(2-x)^2 dx = \frac{7}{3}\]
		which shows that the $y$-intercept of $f$ cannot be above $2$ nor below $-2$ for any $f\in \F_1$.
		Thus, 
			\[|f(x)| \leq |f(0)| + |x| \leq 2 + |x|\]
		for all $f\in \F_1$. By taking the supremum over all $x\in [0,1]$, we conclude that the set
		 $\{f(x) : x\in [0,1], f\in\F_1\}$ is bounded with an upper bound of $M = 3 > 0$. Therefore, $\F_1$ is uniformly bounded.
\end{enumerate}
	
	Therefore, by the Arzela-Ascoli theorem, given that $[0,1]$ is a compact metric space and $\F_1$ is closed, 
	equicontinuous, and uniformly bounded, $\F_1$ is compact. \\
	
\hrule

So, this is a really good problem because it makes you think about the behavior of the functions themselves. Also, things may ``look" like they work out, but then, you can find a counterexample that breaks it. \\

\hrule \vspace{2pt}
\hrule

\break

\section{Sequential Compactness to the Rescue}

\subsection{Problems}

\begin{problem}{\#, Spring 2018} INCLUDE PROBLEM STATEMENT.

\end{problem}

\textbf{Solution:} (a)

(Proof by Contradiction).

Let $0 < \alpha < \beta < 1$ be arbitrary and suppose 
\[d(K_\alpha,\partial K_\beta) = 0.\]

We first show that $K_\alpha$ and $\partial K_\beta$ are compact in $\R^m$. Since $\lim\inf_{|x|\to\infty} f(x) \geq 2$, there exists some $M>0$ in $\R$ such that for any $x\in \R^n$ with $|x|\leq M$, $f(x) < \frac{3}{2}$ (otherwise, 2 would not be the limit inferior). Then, $K_\alpha$ and $\partial K_\beta$ are bounded by $M$ because $f(x) \leq \alpha < \frac{3}{2}$ for all $x\in K_\alpha$ and $f(x) = \beta < \frac{3}{2}$ for all $x\in \partial K_\beta$. Observe that the sets can be written as the pre-image of $f$ under a specific set, i.e.,
\[K_\alpha = f^{-1}((-\infty, \alpha])\]
\[\partial K_\beta = f^{-1}(\{\beta\}).\]
Since $(-\infty, \alpha]$ and $\{\beta\}$ are closed in $\R^m$ (under the Euclidean metric) and $f$ is continuous, the pre-image of closed sets under continuous functions are closed. Therefore, by the Heine-Borel theorem in $\R^m$, each set is compact if and only if they are closed and bounded.

Let $\{x_n\}\subset K_\alpha$ and $\{y_n\}\subset \partial K_\beta$ be sequences such that 
	\[ \lim_{n\to\infty} |x_n - y_n| = 0\]
(which follows by the assumption that $d(K_\alpha,\partial K_\beta) = 0$). Since $K_\alpha$ is compact there exists a subsequence $\{x_{n_k}\}$ of $\{x_n\}$ that converges to $x\in K_\alpha$. Similarly, by the compactness of $\partial K_\beta$ there exists a subsequence $\{y_{n_k}\}$ of $\{y_n\}$ that converges to $y\in \partial K_\beta$. Also, $x\neq y$ because if so, we would have that $x \in \partial K_\beta$ which means that $f(x) = \beta$ and $f(x) \leq \alpha$ with $\alpha < \beta$ and this contradicts our assumption that $f$ is a (continuous) function. So, we can assume that each subsequence converges to distinct points in their respective compact spaces. However, we also find that 
	\[\lim_{k\to\infty} |x_{n_k} - y_{n_k}| = 0,\]
because every subsequence of a convergent sequence converges to the same limit point. Yet, by the properties of convergent sequences, 
\[\lim_{k\to\infty} |x_{n_k} - y_{n_k}| = |x - y| \neq 0\]
which contradicts our assumption that there exists sequences in $K_\alpha$ and $\partial K_\beta$ whose difference converges to 0. 

Therefore, $d(K_\alpha,\partial K_\beta) > 0$ for all $0 < \alpha < \beta < 1$. 


\textbf{Solution:} (b)

FINISH ME. Define a function that is a bunch of spikes that have a constant height and the width of each spike is equal to $1/n$ for all $n\in \N$ or something like this.\\

\hrule

\textbf{Notes:} So, this was a super super hard problem because the notation is kind of aweful. Yet, they are conveying a ``simple" idea. In the first problem, because the limit inferior of the function stays bounded below by 2 as $|x|$ approaches infinity, there are only ``finitely" many values of $x\in \R^m$ that map to values below 2. The choice of the word finite is weird because the real numbers are uncountable. Yet, we can visualize the idea as a bounded subset of real numbers that map to values less than 2 as we approach infinity. Still feels weird. You're tired.\\

\hrule \vspace{2pts}
\hrule

\break

\end{document}